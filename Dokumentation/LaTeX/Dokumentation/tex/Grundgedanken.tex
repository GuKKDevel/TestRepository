\section{Grundlegende Gedanken}

Um die Vorgänge in einer (Hobby-)Imkerei vernünftig modellieren zu können, sind
einige Vorüberlegungen über die anfallenden Arbeiten notwendig. Zum einen ist da
die Tierhaltung, auch Völkerführung oder Betriebsweise genannt, zum anderen gibt
es noch --neben der Honiggewinnung-- die Königinnenzucht, den Verkauf von
Völkern (Ableger/Wirtschaftsvölker), die Propolisgewinnung und die
Wachsgewinnung. \bigskip

In dieser Version des Dokuments geht es primär um die Honigernte, speziell durch
Schleuderung\bigskip.

Für einige dieser Tätigkeiten unterliegt der Imker bestimmten gesetzlichen
Vorschriften und ist daher verpflichtet, in diesen Bereichen seine Handlungen zu
dokumentieren.\medskip

Dazu gibt es zum einen das Bestandsbuch, in dem der Imker sämtliche Bahandlungen
seiner Bienenvölker mit apothekenpflichtigen Medikamente eintragen muß, seit
2022 auch für die nicht apothekenpflichtigen. \medskip


Da der (Hobby-)Imker ein Lebensmittel in Verkehr bringt, wenn er seinen Honig verkauft oder selbst wenn er ihn nur
verschenkt, ist er zum anderen auch verpflichtet, ein sogenanntes Honigbuch zu führen, aus dem
jederzeit ersichtlich ist, was er mit dem Honig gemacht hat, von der Ernte bis
zur Abgabe des Gebindes (Glas o.ä). \bigskip

\section{Honigernte}
Bei der Honigernte durch Schleuderung, Auspressen oder Tropfenlassen werden
Honigwaben von einem oder mehreren Völkern abgeerntet und in ein oder mehrere
Lagergebinde gefüllt. Einen Sonderfall bildet hier der Wabenhonig, der sofort in ein Verkaufsgebinde gefüllt wird.

\subsection {Arten der Honigernte}
\paragraph{Ernte von Wabenhonig}
Bei dieser Art der Honigernte werden nur Waben im Naturbau verwendet. Diese
Waben, die in nicht gedrahteten Rähmchen ausgebaut wurden, werden aus dem
Rähmchen gelöst und dann portionsweise abgeschnitten und direkt in Schälchen
oder sonstige Verkaufsgebinde abgefüllt.
\paragraph{Honigernte durch Pressen}
Die Waben werden aus dem Bienenstock genommen, aus den Rähmchen geschnitten und
in einer Presse ausgepresst, ohne dass sie vorher entdeckelt wurden. Der
ausgepresste Honig wird in einem Sammelgefäß aufgefangen.
\paragraph{Honigernte durch Tropfen lassen}
Die Waben werden aus dem Volk genommen, entdeckelt und zum Austropfen über ein
Sammellgefäß gehängt.
\paragraph{Honigernte durch Schleuderung}
Die Waben werden aus dem Volk entnommen, entdeckelt und in einer Schleuder
ausgeschleudert und in einem Gefäß gesammelt.


\subsection{Ablauf der Honigernte}

Nachdem der Honig nun in den Lagergebinden gesammelt wurde,
werden diese nun gelagert oder der Honig wird nachbereitet.
Nachbereitet wird er, in dem er gerührt wird und somit eine cremige Konsistenz
erlangt oder mehrere Lagergebinde werden gemischt und widerum in ein
Lagergebinde gefüllt bzw. gerührt.

Durch die Abfüllung wird ein Lagergebinde in die Verkaufsgebinde
gefüllt.
\bigskip


\subsection{Nachbereitung für die Honigernte}
 
